Може да видите {\itshape документацията} на кода, генерирана от {\bfseries Doxy\+Gen} \href{https://ivanov1234159.github.io/Calendar/html/}{\tt тук}

{\bfseries Автор\+:} Веселин Иванов

\subsection*{Проект 15\+: Личен календар}

Да се напише програма, реализираща информационна система, която поддържа личен календар, като го записва във файл.

След като приложението отвори даден файл, то трябва да може да извършва посочените по-\/долу операции, в допълнение на общите операции (open, close, save, save as, help и exit)\+:

\begin{quote}
book $<$date$>$ $<$starttime$>$ $<$endtime$>$ $<$name$>$ $<$note$>$ \end{quote}


Запазва час за среща с име $<$name$>$ и коментар $<$note$>$ на дата $<$date$>$ с начален час $<$starttime$>$ и краен час $<$endtime$>$.

\begin{quote}
unbook $<$date$>$ $<$starttime$>$ $<$endtime$>$ \end{quote}


Отменя час за среща на дата $<$date$>$ с начален час $<$starttime$>$ и краен час $<$endtime$>$.

\begin{quote}
agenda $<$date$>$ \end{quote}


Извежда хронологичен списък с всички ангажименти за деня $<$date$>$.

\begin{quote}
change $<$date$>$ $<$starttime$>$ $<$option$>$ $<$newvalue$>$ \end{quote}


$<$option$>$ е едно от date, starttime, enddate, name, note. Задава нова стойност $<$newvalue$>$ на събитието на дата $<$date$>$ с начален час $<$starttime$>$, като при промяна на дата и час се прави проверка дали са коректни и свободни.

\begin{quote}
find $<$string$>$ \end{quote}


Търсене на среща\+: извеждат се данните за всички срещи, в чието име или бележка се съдържа низът $<$string$>$.

\begin{quote}
holiday $<$date$>$ \end{quote}


Датата $<$date$>$ се отбелязва като неработна.

\begin{quote}
busydays $<$from$>$ $<$to$>$ \end{quote}


Извеждане на статистика за натовареност\+: по дадени начална дата $<$from$>$ и крайна дата $<$to$>$ се извежда списък с дните от седмицата, подредени по критерия “брой заети часове”.

\begin{quote}
findslot $<$fromdate$>$ $<$hours$>$ \end{quote}


Намиране на свободно място за среща\+: по дадена дата $<$fromdate$>$ и желана продължителност на срещата $<$hours$>$ търси дата, на която е възможно да се запази такава среща, но само в работни дни и не преди 8 часа или след 17 часа.

\begin{quote}
findslotwith $<$fromdate$>$ $<$hours$>$ $<$calendar$>$ \end{quote}


Намиране на свободно място за среща, синхронизирана с даден календар\+: по дадена дата $<$fromdate$>$ и желана продължителност на срещата $<$hours$>$ търси дата, на която е възможно да се запази такава среща в текущия календар и в календара, записан във файл $<$calendar$>$, но само в работни дни и не преди 8 часа или след 17 часа.

\begin{quote}
merge $<$calendar$>$ \end{quote}


Прехвърля всички събития от календара, записан във файл $<$calendar$>$, в текущия календар. Прехвърлянето да става в диалогов режим така, че ако има конфликт на събития потребителят да има възможност да избере кое събитие да остане и кое да се премести в друг ден и час.

{\bfseries Пример\+:}

потребителят се е записал на спорт и е получил файл, който съдържа календар с всички тренировки и спортни събития. Той иска да прехвърли всички спортни събития в календара си.

\subsection*{Бонуси\+:}

командите findslotwith и merge да поддържат повече от един календар.

\subsection*{}

\section*{Външни източници}


\begin{DoxyItemize}
\item \href{https://github.com/onqtam/doctest}{\tt doctest} -\/ за тестването на проекта (програмата) 
\end{DoxyItemize}